\section{Mikrocontroller}
\subsection*{Schalenmodell}
\begin{center}
    \includegraphics[width=0.65\textwidth]{images/schalenmodell.png}    
\end{center}
\begin{itemize} 
    \item \textbf{Prozessorkern:}
    \begin{itemize}
        \item Auf die Aufgabe zugeschnitten
    \end{itemize}
    \item \textbf{Scheib-/Lesespeicher:}
    \begin{itemize}
        \item Speicherung von Daten, unterschiedliche Größe je nach Anwendung
    \end{itemize}
    \item \textbf{Festwertspeicher:}
    \begin{itemize}
        \item Speicherung von Programmen und Konstanten
        \item Je nach Anwendung andere Größe und Typ (ROM, PROM, \dots)
    \end{itemize}
    \item \textbf{Serielle und parallele Ein-/Ausgabekanäle:}
    \begin{itemize}
        \item Digitale Schnittstelle
        \item Asynchrone oder synchrone Datenübertragung
    \end{itemize}
    \item \textbf{AD/DA-Wandler:}
    \begin{itemize}
        \item Analoge Schnittstelle
        \item Wichtige Faktoren: Auflösung, Wandlungszeit, Spannungsbereich und Wandlungsfehler
    \end{itemize}
    \item \textbf{Zähler \& Zeitgeber:}
    \begin{itemize}
        \item Messung von Zeiten, Zählen von Ereignissen, Pulsweitenmodulation, \dots
        \item Wichtig für viele Echtzeitanwendungen
    \end{itemize}
    \item \textbf{Watchdog:}
    \begin{itemize}
        \item Überwacht Programm auf in regelmäßigen Abständen gesendete Lebenszeichen
        \item Bei Ausbleiben des Lebenszeichen $\Rightarrow$ Annahme eines Fehlers und Reset
    \end{itemize}
    \item \textbf{Echtzeitkanäle:}
    \begin{itemize}
        \item Echtzeiterweiterung: parallelen E/A-Kanal mit Zeitgeber koppeln
        \item Ein-/Ausgabezeitpunkt wird durch Hardware gesteuert $\Rightarrow$ Jitter vermieden
    \end{itemize}
    \item \textbf{Interrupts:}
    \begin{itemize}
        \item Reaktion auf Ereignisse, z.B. durch Eingangssignale, Zählerm Zeitgeber, \dots
    \end{itemize}
    \item \textbf{DMA (Direct Memory Access):}
    \begin{itemize}
        \item Direkter Datentransfer zwischen Peripherie und Speicher ohne Beteiligung des Prozessorkerns
        \item Prozessorkern wird entlastet und kann währenddessen andere Aktivitäten ausführen
    \end{itemize}
    \item \textbf{Ruhebetrieb:}
    \begin{itemize}
        \item Für Batterie- bzw. Akkubetrieb wichtig
    \end{itemize}
    \item \textbf{Erweiterungsbus:}
    \begin{itemize}
        \item Anschluss externen Komponenten
    \end{itemize}
\end{itemize}
    
\subsection*{Mikrocontroller vs. System-on-Chip}
\begin{itemize}
    \item \textbf{Mikrocontroller:}
    \begin{itemize}
        \item Bausteine auf dem Chip: CPU, Speicher, E/A-Schnittstelle
        \item Erweiterung durch weitere externe Bausteine $\Rightarrow$ anwendungsspezifisches System
    \end{itemize}
    \item \textbf{System-on-Chip:}
    \begin{itemize}
        \item Weiterentwicklung der Mikrocontroller-Idee
        \item Weitere Bausteine auf Chip integriert $\Rightarrow$ anwendungsspezifisches System
    \end{itemize}
\end{itemize}

\section{Energiespartechniken}
\subsection*{Formeln}
\begin{itemize}
    \item \textbf{Leistungsaufnahme $P$:}
    \begin{itemize}
        \item statischer Anteil $P_s$ (Leckströme)
        \item dynamischer Anteil $P_d$ (abhägig von Taktfrequenz $F$)
        \item $P = P_s + P_d = c_s + c_d \cdot F$
    \end{itemize}
    \item \textbf{Ausführungsdauer $T_a$ (umgekehrt proportional zur Taktfrequenz)}
    \begin{itemize}
        \item mit Proportionalitätsfaktor $c_a$ folgt $T_a = \frac{c_a}{F}$
    \end{itemize}
    \item \textbf{Energieverbrauch  $E_a$ (bei konstanter Leistungsaufnahme)}
    \begin{itemize}
        \item $E_a = P \cdot T_a = P \cdot \frac{c_a}{F} = (c_s + c_d \cdot F) \cdot \frac{c_a}{F} = \frac{c_s \cdot c_a}{F} + c_d \cdot c_a$
    \end{itemize}
\end{itemize}

\subsection*{Energiespartechniken}
\begin{itemize}
    \item \textbf{Taktfrequenzreduktion:}
    \begin{itemize}
        \item Reduktion der Taktfrequenz senkt die Leistungsaufnahme
              (dynamischer Anteil $P_d = c_d \cdot F$ sinkt)
        \item Achtung: Energieverbrauch steigt dabei aufgrund des statischen Anteils
              ($\Delta E_a = \frac{1}{9} \cdot \frac{c_s \cdot c_a}{F} > 0$)
    \end{itemize}

    \item \textbf{Komponentenabschaltung:}
    \begin{itemize}
        \item Busschnittstelle abschalten, wenn RISC-Operationen auf
              internen Registern arbeiten (Load-/Store-Architektur)
        \item Nicht benötigte Leitungen der 64-Bit-Busschnittstelle abschalten
              (bei Verwendung schmaler Datentypen)
        \item Teile der ALU und der internen Datenpfade abschalten
              (bei schmalen Datentypen)
        \item Allgemein: nicht benötigte Komponenten abschalten
    \end{itemize}

    \item \textbf{Spannungsreduktion:}
    \begin{itemize}
        \item SRAM-Versorgungsspannung so weit reduzieren,
              dass die Daten gerade noch aufrechterhalten bleiben
    \end{itemize}

    \item \textbf{Mindesttaktfrequenz:}
    \begin{itemize}
        \item Bei dynamischem Steuerwerk (kein statisches Steuerwerk)
              muss eine Mindesttaktfrequenz eingehalten werden
    \end{itemize}

    \item \textbf{Erhöhung der Codedichte durch CISC-Architektur:}
    \begin{itemize}
        \item CISC besitzt umfangreichen Befehlssatz $\Rightarrow$
              höhere Codedichte erreichbar
        \item Höhere Codedichte reduziert Speicherzugriffe
              $\Rightarrow$ weniger Energieverbrauch beim Speicher
        \item RISC besitzt reduzierten Befehlssatz $\Rightarrow$
              geringere Codedichte
    \end{itemize}
\end{itemize}
