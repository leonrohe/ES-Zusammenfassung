\section{Klassifizierung von Echtzeitsystemen}

\textbf{Harte Zeitschranken}
\begin{itemize}
    \item System muss Zeitschranke einhalten, bei Überschreiten entsteht Schade
    \item \textit{z.B. Radarabtastung bei Flugzeug}
\end{itemize}

\textbf{Feste Zeitschranken}
\begin{itemize}
    \item Kein Schaden bei Überschreitung, Ergebnis aber wertlos
    \item[$\rightarrow$] Überschreitung führt zum Abbruch der ausgeführten Aktion
    \item \textit{z.B. Anzeige der Fahrtzeiten des öffentlichen Nahverkehrs}
\end{itemize}

\textbf{Weiche Zeitschranken}
\begin{itemize}
    \item Zeitschranken eher als Richtwert und bis zu gewissem Grad überschreitbar
    \item \textit{z.B. Temperaturregelung eines Kühlschrankes}
\end{itemize} 

\section{Endliche Automaten}
\subsection*{Funktionen}
\begin{itemize}
    \item modellieren Verhalten eines Systems
    \item beschreiben aktionen, Zustände und Zustandsübergänge
\end{itemize}

\subsection*{Moore vs. Mealy}
\begin{table}[h]
    \begin{tabular}{@{} m{6cm} m{6cm} @{}}
        \toprule
        \textbf{Moore-Automat} & \textbf{Mealy-Automat} \\
        \midrule
        Eingabealphabet: $\Sigma$ & Eingabealphabet: $\Sigma$ \\
        Übergangsfunktion: $\delta: Z \times \Sigma \to Z$ & Übergangsfunktion: $\delta: Z \times \Sigma \to Z$ \\
        Ausgabefunktion: $\lambda: Z \to \Delta$ & Ausgabefunktion: $\lambda: Z \times \Sigma \to \Delta$ \\
        Startzustand: $z_0 \in Z$ & Startzustand: $z_0 \in Z$ \\
        \bottomrule
    \end{tabular}
\end{table}

\section{Petrinetze}
\subsection*{Funktionen}
\begin{itemize}
    \item Modellierung nebenläufiger Prozesse
    \item Petri-Netz als 6-Tupel $(S, T, F, K, W, m_{0})$
    \begin{itemize}
        \item \textbf{S:} nichtleere, endliche Menge von Stellen
        \item \textbf{T:} nichtleere, endliche Menge von Transitionen
        \item \textbf{F:} nichtleere, endliche Menge von Kanten; F $\in (S \times T)$
        \item \textbf{K:} Kapazitäten der Stellen
        \item \textbf{W:} Kosten von Kanten
        \item \textbf{$m_0$} Startmarkierung des Netzes
    \end{itemize}
    \item Das 3-Tupel $(S, T, F)$ ist ein bipartiter und gerichteter Graph
\end{itemize}

\textbf{Beispiel Petrinetz}\\
\includegraphics[width=\textwidth]{images/petrinetz.png}

\subsection*{Steuerung vs. Regelung}
\begin{itemize}
    \item \textbf{Steuerung:} Steuerung von Strecke durch Steuerglied, ohne unvorhergesehenes Streckenverhalten zu berücksichtigen
    \item \textbf{Regelung:} Steuerung von Strecke durch Regler unter Berücksichtigung des Streckenverhaltens (Rückkopplung), um ein Regelziel (Sollwert) zu erreichen
\end{itemize}

\subsection*{Übertragungsfunktionen}
\begin{itemize}
    \item \textbf{Reihenschaltung:} \\
    \includegraphics[width=0.5\linewidth]{images/reihe_vorher.png}
    \includegraphics[width=0.5\linewidth]{images/reihe_nachher.png}
    \item \textbf{Parallelschaltung:} \\
    \includegraphics[width=0.5\linewidth]{images/parallel_vorher.png}
    \includegraphics[width=0.5\linewidth]{images/parallel_nachher.png}
    \item \textbf{Gegenkopplung:} \\
    \includegraphics[width=0.5\linewidth]{images/gegen_vorher.png}
    \includegraphics[width=0.5\linewidth]{images/gegen_nachher.png}
\end{itemize}