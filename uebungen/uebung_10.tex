\section{Tasksynchronisation}

\subsection*{Verklemmungen}

\begin{itemize}
    \setlength{\itemsep}{4pt}

    \item \textbf{Deadlock} -- zwei oder mehr Tasks blockieren sich gegenseitig:
    \begin{itemize}
        \setlength{\itemsep}{2pt}
        \item Jede Task wartet auf ein Betriebsmittel, das von einer anderen Task gehalten wird
        \item Kein Fortschreiten der beteiligten Tasks möglich
    \end{itemize}

    \item \textbf{Livelock} -- eine Task wird durch andere Tasks ständig an der Ausführung gehindert:
    \begin{itemize}
        \setlength{\itemsep}{2pt}
        \item Sonderfall: Livelock durch \textbf{Prioritätsinversion} -- hochpriore Task muss auf niederpriore Task warten (z.B.\ wegen gemeinsamer Ressource)
    \end{itemize}

\end{itemize}

\subsection*{Synchronisationsmittel}

\begin{itemize}
    \setlength{\itemsep}{4pt}

    \item \textbf{Semaphor} -- Zählvariable zur Ressourcenverwaltung:
    \begin{itemize}
        \setlength{\itemsep}{2pt}
        \item Initialisierung beliebig; Maximalwert beliebig
        \item \textit{Passieren}: Dekrementieren; blockieren, falls Wert $< 0$
        \item \textit{Verlassen}: Inkrementieren; ggf.\ wartende Task freigeben
        \item Passieren und Verlassen müssen nicht von derselben Task durchgeführt werden $\Rightarrow$ Reihenfolgensynchronisation möglich
    \end{itemize}

    \item \textbf{Mutex} -- binäres Sperrmittel mit Eigentümerkonzept:
    \begin{itemize}
        \setlength{\itemsep}{2pt}
        \item Initial immer frei; Maximalwert 1 -- höchstens eine Task gleichzeitig im kritischen Abschnitt
        \item \textit{Sperren}: Blockieren bis frei, dann auf belegt setzen
        \item \textit{Freigeben}: Nur durch die Task möglich, die den Mutex zuvor gesperrt hat (\textit{ownership})
        \item Häufig \textbf{wiedereintrittsfähig} (\textit{reentrant}): Eigentümer darf Mutex mehrfach sperren; Freigabe erst durch letzte Entsperrung
        \item Keine Reihenfolgensynchronisation möglich; Erzeuger-/Verbraucher-Muster nicht realisierbar
    \end{itemize}

\end{itemize}

\section{Echtzeitbetriebssysteme}

\subsection*{Aufgaben}

\begin{itemize}
    \setlength{\itemsep}{4pt}

    \item \textbf{Standardbetriebssystem:}
    \begin{itemize}
        \setlength{\itemsep}{2pt}
        \item Taskverwaltung (Prozessorzuteilung)
        \item Betriebsmittelverwaltung (Speicher, I/O)
        \item Interprozesskommunikation
        \item Synchronisation
        \item Schutzmaßnahmen
    \end{itemize}

    \item \textbf{Zusätzlich im Echtzeitbetriebssystem:}
    \begin{itemize}
        \setlength{\itemsep}{2pt}
        \item Wahrung der Rechtzeitigkeit und Gleichzeitigkeit -- Daten müssen rechtzeitig abgeholt, verarbeitet und ausgegeben werden
        \item Wahrung der Verfügbarkeit -- keine unvorhersehbaren Unterbrechungen des Betriebs
    \end{itemize}

\end{itemize}

\subsection*{Kernarchitekturen}

\begin{itemize}
    \setlength{\itemsep}{4pt}

    \item \textbf{Makrokernbetriebssystem} -- umfangreiche Funktionalität im \textit{kernel mode}:
    \begin{itemize}
        \setlength{\itemsep}{2pt}
        \item Taskverwaltung inkl.\ Scheduling, Speicherverwaltung, IPC, Synchronisation, Treiber
        \item Moderner Ansatz: hierarchische Schichten statt monolithischem Block
        \item Nachteil: Schlechte zeitliche Vorhersagbarkeit durch lange kritische Bereiche
    \end{itemize}

    \item \textbf{Mikrokernbetriebssystem} -- schlanker Kern im \textit{kernel mode}:
    \begin{itemize}
        \setlength{\itemsep}{2pt}
        \item Nur IPC, elementare Taskverwaltung (kein Scheduling), Synchronisation und Speicherverwaltung im Kern
        \item Alle übrigen Funktionen als Module im \textit{user mode}
        \item \textbf{Vorteile:} Hohe Anpass- und Skalierbarkeit, einfache Portierbarkeit, bessere zeitliche Vorhersagbarkeit, ggf.\ formale Verifikation möglich (z.B.\ seL4)
        \item \textbf{Nachteil:} Häufige Kontextwechsel zwischen \textit{user mode} und \textit{kernel mode} -- IPC und Kontextwechsel müssen daher sehr schnell sein
    \end{itemize}

\end{itemize}
