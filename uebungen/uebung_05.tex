\section{AD-Wandlung}
\subsection*{Formeln}
\begin{itemize}
    \item $U_{LSB} = \frac{U_{max}-U_{\text{min}}}{2^n}$
    \item Quantisierungsfehler: \\
    $U_{QF} = \text{diskretisierte Spannung} - \text{tatsächliche Spannung} = Z \cdot U_{LSB} - U_{in}$ \\
    $\overline{U_{QF}} = -0.5U_{LSB}$ solange $Z$ abgerundet wird
\end{itemize}

\subsection*{Wägerverfahren}
\begin{center}
    \includegraphics[width=0.49\textwidth]{images/waegerverfahren.png}
    \includegraphics[width=0.49\textwidth]{images/waegerverfahren_bsp.png}
\end{center}
\begin{itemize}
    \item Verwendet sukzessive Approximation:
    \begin{itemize}
        \item Zu Beginn: $Z = 100...0_2$, also $U_g(Z) = 2^{n-1} \cdot U_{LSB}$
        \item $U_g(Z) > U_{in}$: \\
        Aktuelles Bit auf 0 setzen, Bit mit nächstkleinerer Wertigkeit auf 1 setzen
        \item $U_g(Z) \leq U_{in}$ \\
        Bit mit nächstkleinerer Wertigkeit auf 1 setzen
        \item Für alle n-Bits werden n Schritte benötigt
    \end{itemize}
\end{itemize}

\subsection*{Abtast-/Halteglied}
\begin{center}
    \includegraphics[width=0.6\textwidth]{images/halteglied.png}
\end{center}

\begin{itemize}
    \item Hält das Eingabesignal über die Dauer der Wandlung konstant $\Rightarrow$ verhindert Signaländerung während der n Wandlunsschritte
    \item Schalter S wird bei Abtastung geschlossen
    \begin{itemize}
        \item Operationsverstärker entkoppelt das Abtastglied vom Eingang
        \item Kondensator wird mit $U_e$ aufgeladen
    \end{itemize}
    \item Danach wird Schalter S wieder geöffnet
    \begin{itemize}
        \item Kondensator bleibt geladen, da $U_a$ vom Operationsverstärker entkoppelt wird
        \item Signal kann abgetastet werden
    \end{itemize}
\end{itemize}